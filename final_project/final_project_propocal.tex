\documentclass{acm_proc_article-sp}
\begin{document}
\title{Final project proposal of parallel programming}
\subtitle{[Free collision in physics engine]}
\numberofauthors{2}
\author{
% 1st. author
\alignauthor
Yi-Feng Chen\\
        \affaddr{Student ID: 0116223}\\
        \email{joy1357924680@gmail.com}
% 2nd. author
\alignauthor
Jin-Xian Xu\\
        \affaddr{Student ID: 0216025}\\
        \email{alexhhxela@gmail.com}
}
\date{4 November 2014}
\maketitle
\section{Introduction}
A physics engine is computer software that provides an approximate simulation of certain physical systems, such as rigid body dynamics (including collision detection), soft body dynamics, and fluid dynamics, of use in the domains of computer graphics, video games and film. Their main uses are in video games (typically as middleware), in which case the simulations are in real-time. The term is sometimes used more generally to describe any software system for simulating physical phenomena, such as high-performance scientific simulation.

\section{Statement of the problem}
Open Dynameics Engine, a.k.a ODE, is an open source, high performance library for simulating rigid body dynamics, And we want to improve free collision part in ODE.

\section{Proposed approaches}
We want to increase the collision part of ODE.

\section{Language selection}
We'll use C++ language, since it's our most familiar language. The parallel librays in C/C++ are also complete.
And maybe include some python code, since the open dynamics engine project's source code include python code.

\section{Related work}
\subsection{Open Dynamics Engine}
ODE is an open source, high performance library for simulating rigid body dynamics. It is fully featured, stable, mature and platform independent with an easy to use C/C++ API. It has advanced joint types and integrated collision detection with friction. ODE is useful for simulating vehicles, objects in virtual reality environments and virtual creatures. It is currently used in many computer games, 3D authoring tools and simulation tools. 
\subsection{Bullet}
Bullet is a physics engine which simulates collision detection, soft and rigid body dynamics. Bullet has been used in video games as well as for visual effects in movies. Erwin Coumans, its main author, worked for Sony Computer Entertainment US R&D from 2003 until 2010, for AMD until 2014, and he now works for Google.\\
The Bullet physics library is free and open-source software subject to the terms of the zlib License.

\section{Statement of expected results}
We want to improve free collision part in ODE, and want to let it be scalable.

\section{Timetable}
Octorber 24, 1014 - Octorber 28, 2014: Brainstroming.\\
Octorber 28, 2014 - November 1, 2014: Deside project.\\
November 1, 2014 - November 3, 2014: Collect information.\\
November 3, 2014 - November 4, 2014: Finish project proposal.\\
November 4, 2014 - November 20, 2014: Trace source code.\\
Novemver 20, 2014 - January 10, 2015: Finish source code.\\
January 15, 2015 - January 20, 2015: Finish final report.\\

\section{Reference}
Wikipedia\\
Open Dynamics Engine

\end{document}
